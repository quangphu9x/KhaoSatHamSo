\documentclass[preview,14pt]{standalone}
\usepackage{amsmath}
\usepackage{tikz}
\usepackage[utf8]{vietnam}
\usepackage{graphicx}
\usepackage{tkz-tab}
\begin{document}
\textbf{Đề bài:} Viết phương trình tiếp tuyến với đồ thị (C): $y = 3\,{x}^{3}-7\,x+2
$  vuông góc với đường thẳng $d_1$: $2\,x+3\,y=1
$

\textbf{Lời giải:}

Gọi tiếp tuyến cần tìm là delta

\textbf{Bước 1:} delta vuông góc với $d_1$, tính được hệ số góc $k$

$k = $$3/2
$

\textbf{Bước 2:} Từ $f(x)$, suy ra hệ số góc của delta: $kx = f'(x)$

$kx = $$9\,{x}^{2}-7
$

\textbf{Bước 3:} Giải phương trình $kx = k$, ta được hoành độ tiếp điểm $x_0$

$x_0 = $$[-1/6\,\sqrt {34},1/6\,\sqrt {34}]
$

\textbf{Bước 4:} Biết được hoành độ tiếp điểm $x_0$, thế vào phương trình $f(x)$ tính được tung độ tiếp điểm $y_0$, Từ đó tìm được tiếp điểm $M(x_0, y_0)$)

$y_0 = $$[{\frac {25\,\sqrt {34}}{36}}+2,-{\frac {25\,\sqrt {34}}{36}}+2]
$

$M = $$[[-1/6\,\sqrt {34},{\frac {25\,\sqrt {34}}{36}}+2],[1/6\,\sqrt {34},-{
\frac {25\,\sqrt {34}}{36}}+2]]
$

\textbf{Bước 5:} delta nhận $M$ làm tiếp điểm, có hệ số góc $k$, ta viết được Phương trình tiếp tuyến

$pttt = $$[3/2\,x+{\frac {17\,\sqrt {34}}{18}}+2,3/2\,x-{\frac {17\,\sqrt {34}}{
18}}+2]
$


\end{document} 